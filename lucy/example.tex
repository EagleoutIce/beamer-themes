\documentclass[aspectratio=169,usepdftitle=true,t]{beamer}

\usepackage[T1]{fontenc}
\usepackage[utf8]{inputenc}
\usepackage{microtype}
\usepackage[english,main=ngerman]{babel}

\usetheme{lucy}

\title{Example Presentation}
\subtitle{This is madness}
\institute{Cute-Dispute-Institute \(\circ\) University Ulm}

\date{14, August 2020}

\author{Florian Sihler}

% just an example command
\newcommand\twosplit[3][t]{%
    \begin{columns}[#1]
    \begin{column}{0.475\linewidth}
        #2
    \end{column}\hfill
    \begin{column}{0.475\linewidth}
        #3
    \end{column}
    \end{columns}
}

\usepackage[backend=bibtex8,style=alphabetic]{biblatex}
\addbibresource{../example.bib}

\begin{document}

\begin{frame}[c]
    \titlepage
\end{frame}

\begin{frame}[c]{The table of contents}
    \begin{center}
        \twosplit{\tableofcontents[sections={1-2}]}
            {\tableofcontents[sections={3-4}]}
    \end{center}
\end{frame}


\section{Theoretische Grundlagen}
\subsection{Der Algorithmusbegriff}
\begin{frame}{Was ist ein Algorithmus?}
    \begin{definition}[Algorithmus]
        Ein \emph{Algorithmus} ist eine \emph{eindeutige Handlungsvorschrift} zur Lösung eines Problems.
    \end{definition}
    \begin{itemize}
        \item Es existieren verschiedene Darstellungsformen: \begin{itemize}
            \item Textuell (1. Tue dies, 2. Tue das, 3. Falls \(X\)\ldots)
            \item Grafisch (Instruktionsabläufe, Graphen, Grafiken, \ldots)
            \item Pseudocode
        \end{itemize}
        \item Klassische Alltagsbeispiele: (Koch-)Rezepte, Gebrauchsanleitungen, \ldots
    \end{itemize}
\end{frame}

\subsection{Algorithmen analysieren}
\begin{frame}{Algorithmuseigenschaften}
    \begin{itemize}
        \item Zu Beginn gilt es einige Begriffe zu klären: \begin{description}[Elementaroperation]
            \item[Prozess] Die Ausführung der Schritte eines Algorithmus
            \item[Prozessor] Der Ausführende (Mensch, Computer, \ldots)
            \item[Elementaroperation] Eine einzelne, eindeutige Handlung. 
        \end{description}
        \item Ein Algorithmus besitzt einige Eigenschaften: \begin{description}[Ausführbarkeit]
            \item[Ausführbarkeit] Die Anleitung muss ausführ- und reproduzierbar sein.
            \item[Endlichkeit] Der gesamte Algorithmus muss mit endlich viel Text beschrieben werden können.
            \item[Termination] Der Algorithmus muss nach endlich vielen Schritten zum Ende kommen.
            \item[Verarbeitung] Ein Algorithmus erhält Eingabeobjekte \(E\), hält lokale Daten \(D\) und erzeugt Ausgabedaten \(A\).
        \end{description}
    \end{itemize}
\end{frame}

\begin{frame}[c]{Eigenschaften für die Analyse}
    \twosplit{%
    \begin{block}{Korrektheit}
        Ein Algorithmus ist korrekt, wenn er für jede (definierte) Eingabe, die korrekte Ausgabe erzeugt.
    \end{block}
    \begin{block}{Partielle Korrektheit}
        Wenn der Algorithmus terminiert, ist er korrekt.
    \end{block}
    }{%
    \begin{block}{Robust}
        Ein Algorithmus ist (maximal) robust, wenn er falsche Eingabe erkennen und abfangen kann.
    \end{block}
    \begin{block}{Totale Korrektheit}
        Der Algorithmus ist partiell korrekt und terminiert.
    \end{block}
    }
\end{frame}

\begin{frame}{Eigenschaften für die Analyse}
    \twosplit{%
    \begin{block}{Determinismus}
        Der Ablauf des Algorithmus ist \emph{eindeutig}. Für jede Anweisung folgt bei gleichen Vorraussetzungen dieselbe.
    \end{block}
    }{%
    \begin{block}{Determiniertheit}
        Die selben Eingabedaten erzeugen immer die selben Ausgabedaten.
    \end{block}
    }\bigskip
    \begin{itemize}
        \item Die Begriffe sind zu unterscheiden!
        \item So können bei einem Algorithmus verschiedene Wege zum selben Ziel führen.
        \item (Sinnfreies) Beispiel: Der Algorithmus kann jedes Element eines Arrays quadrieren. Die Reihenfolge wählt er zufällig aus!
        \item Dieser Algorithmus \emph{determiniert}, ist aber \emph{nicht-deterministisch}!
    \end{itemize}
\end{frame}


\section{Another example}
\subsection{This is an example}

\begin{frame}{Mega-Example 1}
    Hello \cite{knuth-fa}!
\end{frame}

\begin{frame}{Mega-Example 2}
    World
\end{frame}

\subsection{This is an example, 2}

\begin{frame}{Mega-Example 1}
    Hello
\end{frame}

\begin{frame}{Mega-Example 2}
    World \cite{knuth-web}!
\end{frame}

\section{Talk the talker}
\subsection{This is an example}

\begin{frame}{Mega-Example 1}
    Hello \cite{dirac}!
\end{frame}

\begin{frame}{Mega-Example 2}
    World as seen in \cite{einstein}.
\end{frame}

\subsection{This is an example, 2}

\begin{frame}{Mega-Example 1}
    Hello
\end{frame}

\begin{frame}{Mega-Example 2}
    World
\end{frame}


\section{Conclusion}
\subsection{This is an example}

\begin{frame}{Mega-Example 1}
    Hello
\end{frame}

\begin{frame}{Mega-Example 2}
    World
\end{frame}

\subsection{This is an example, 2}

\begin{frame}{Mega-Example 1}
    Hello
\end{frame}

\begin{frame}{Mega-Example 2}
    World
\end{frame}

\begin{frame}{Bibliography}
    \printbibliography
\end{frame}

\end{document}