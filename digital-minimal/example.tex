\documentclass[aspectratio=169,usepdftitle=true]{beamer}

\usepackage[T1]{fontenc}
\usepackage[utf8]{inputenc}
\usepackage{microtype}
\usepackage[english,main=ngerman]{babel}
\usepackage{lipsum}

\usetheme{digital-minimal}

\title{Example Presentation}
\subtitle{This is madness}

\date[14.08.2020]{14, August 2020}

\author[F.~Sihler]{Florian Sihler}

\usepackage[backend=bibtex8,style=alphabetic]{biblatex}
\addbibresource{example.bib}
\begin{document}

\section{Theoretische Grundlagen}
\subsection{Der Algorithmusbegriff}
\begin{frame}{Was ist ein Algorithmus?}
    \begin{itemize}
        \item Es existieren verschiedene Darstellungsformen: \begin{itemize}
            \item Textuell (1. Tue dies, 2. Tue das, 3. Falls \(X\)\ldots)
            \item Grafisch (Instruktionsabläufe, Graphen, Grafiken, \ldots)
            \item Pseudocode
        \end{itemize}
        \item Klassische Alltagsbeispiele: (Koch-)Rezepte, Gebrauchsanleitungen, \ldots
    \end{itemize}
\end{frame}

\subsection{Algorithmen analysieren}
\begin{frame}{Algorithmuseigenschaften}
    \begin{itemize}[<+(1)->]
        \item Zu Beginn gilt es einige Begriffe zu klären: \begin{description}[Elementaroperation]
            \item[Prozess] Die Ausführung der Schritte eines Algorithmus
            \item[Prozessor] Der Ausführende (Mensch, Computer, \ldots)
            \item[Elementaroperation] Eine einzelne, eindeutige Handlung.
        \end{description}
    \end{itemize}
\end{frame}

\begin{frame}{Eigenschaften für die Analyse}
    \begin{itemize}
        \item Die Begriffe sind zu unterscheiden!
        \item So können bei einem Algorithmus verschiedene Wege zum selben Ziel führen.
        \item (Sinnfreies) Beispiel: Der Algorithmus kann jedes Element eines Arrays quadrieren. Die Reihenfolge wählt er zufällig aus!
        \item Dieser Algorithmus \emph{determiniert}, ist aber \emph{nicht-deterministisch}!
    \end{itemize}
\end{frame}


\section{Another example}
\subsection{This is an example}

\begin{frame}{Mega-Example 1}
    Hello \cite{knuth-fa}!
\end{frame}

\begin{frame}{Mega-Example 2}
    World
\end{frame}

\subsection{This is an example, 2}

\begin{frame}{Mega-Example 1}
    Hello
\end{frame}

\begin{frame}{Mega-Example 2}
    World \cite{knuth-web}!
\end{frame}

\section{Talk the talker}
\subsection{This is an example}

\begin{frame}{Mega-Example 1}
    Hello \cite{dirac}!
\end{frame}

\begin{frame}{Mega-Example 2}
    World as seen in \cite{einstein}.
\end{frame}

\subsection{This is an example, 2}

\begin{frame}{Mega-Example 1}
    Hello
\end{frame}

\begin{frame}{Mega-Example 2}
    World
\end{frame}


\section{Conclusion}
\subsection{This is an example}

\begin{frame}{Mega-Example 1}
    Hello
\end{frame}

\begin{frame}{Mega-Example 2}
    World
\end{frame}

\subsection{This is an example, 2}

\begin{frame}{Mega-Example 1}
    Hello
\end{frame}

\begin{frame}{Mega-Example 2}
    World
\end{frame}

\begin{frame}{Bibliography}
    \printbibliography
\end{frame}

\end{document}